\documentclass[12pt]{article}
\usepackage{graphicx}
\usepackage{amsfonts}
\usepackage{amsmath}
\usepackage{fullpage}
\usepackage[utf8x]{inputenc}
\makeatletter
\usepackage{cite}
\usepackage{hyperref}
\@namedef{opt@inputenc.sty}{utf8}
\makeatother
\usepackage{CJKutf8}
\usepackage{multirow}
\usepackage{array}
\newenvironment{conditions}
  {\par\vspace{\abovedisplayskip}\noindent\begin{tabular}{>{$}l<{$} @{${}={}$} l}}
  {\end{tabular}\par\vspace{\belowdisplayskip}}
\begin{document}
\title{
\begin{flushright}
\includegraphics[width=2.5cm]{logoUHi.jpg}\\
{\small
Data Analytics\\
Stiftung Universit{\"a}t Hildesheim\\
Marienburger Platz 22\\
31141 Hildesheim\\
Prof. Dr. Dr. Lars Schmidt-Thieme\\
}
\end{flushright}
\bigskip
\begin{center}
Thesis\\
Unsupervised Real-Time Time-Series Anomaly Detection\\
\end{center}
}
\author{Abdul Rehman Liaqat}
\date{271336, Liaqat@uni-hidesheim.de}
\maketitle

\newpage

\begin{abstract}
Anomaly detection is a crucial task for machine learning due to wide-spread usage and type. In particular, it is worth noting that most data arising in industrial setups are of a streaming nature, thus restricting the range of standard anomaly detection tools. This thesis will identify the potential approaches to learn the identification of abnormal behavior from large-scale streaming data. An empirical comparison of state-of-the-art methods will to be extended by a novel technical contribution. In this thesis, the focus is particularly on streaming time-series Anomaly Detection which changes in nature with time and novel contribution will especially try to target this dynamic nature of time-series.
\end{abstract}
\newpage
\tableofcontents
\newpage
\section{Introduction}
\newpage
\subsection{Motivation}
\newpage
\subsection{Objective}
\newpage
\section{Related Work}
\newpage
\section{Unsupervised Anomaly detection with recency}
\newpage
\section{Experiments}
\newpage
\subsection{Data}
\subsubsection{Numenta Anomaly Benchmark (NAb)}
\newpage
\section{Execution and Results}
\newpage
\section{Discussion}
\newpage
\section{Experiment Infrastructure}
\newpage
\subsection{Experiment Management using MLflow}
\newpage
\subsection{Parallel execution using Docker}
\newpage
\section{References}
\begingroup
\nocite{*}
\renewcommand{\section}[2]{}
\bibliographystyle{IEEEtrans}`
\bibliography{third}
\endgroup
\end{document}
