\documentclass[12pt]{article}
\usepackage{graphicx}
\usepackage{amsfonts}
\usepackage{amsmath}
\usepackage{fullpage}
\usepackage[utf8x]{inputenc}
\makeatletter
\usepackage{cite}
\usepackage{hyperref}
\@namedef{opt@inputenc.sty}{utf8}
\makeatother
\usepackage{url}
\usepackage{CJKutf8}
\usepackage{multirow}
\usepackage{array}
\newenvironment{conditions}
  {\par\vspace{\abovedisplayskip}\noindent\begin{tabular}{>{$}l<{$} @{${}={}$} l}}
  {\end{tabular}\par\vspace{\belowdisplayskip}}
\begin{document}
\title{
\begin{flushright}
\includegraphics[width=2.5cm]{logoUHi.jpg}\\
{\small
Data Analytics\\
Stiftung Universit{\"a}t Hildesheim\\
Marienburger Platz 22\\
31141 Hildesheim\\
Prof. Dr. Dr. Lars Schmidt-Thieme\\
}
\end{flushright}
\bigskip
\begin{center}
Thesis\\
Unsupervised Real-Time Time-Series Anomaly Detection\\
\end{center}
}
\author{Abdul Rehman Liaqat}
\date{271336, Liaqat@uni-hidesheim.de}
\maketitle

\newpage

\begin{abstract}
Anomaly detection is a crucial task for machine learning due to wide-spread usage and type. In particular, it is worth noting that most data arising in industrial setups are of a streaming nature, thus restricting the range of standard anomaly detection tools. This thesis will identify the potential approaches to learn the identification of abnormal behavior from large-scale streaming data. An empirical comparison of state-of-the-art methods will to be extended by a novel technical contribution. In this thesis, the focus is particularly on streaming time-series Anomaly Detection which changes in nature with time and novel contribution will especially try to target this dynamic nature of time-series.
\end{abstract}
\newpage
\tableofcontents
\newpage
\section{Introduction}
\newpage
\subsection{Motivation}
\newpage
\subsection{Objective}
\newpage
\section{Related Work and State of the art}
\newpage
\section{Benchmarks}
\newpage
\subsection{Autoencoder based models}
\subsubsection{Fully connected layers}
\subsubsection{Fully convolution layers}
\subsubsection{LSTM based}
\newpage
\subsection{Prediction based models}
\subsubsection{Fully connected layers}
\subsubsection{Fully convolution layers}
\subsubsection{LSTM based}
\newpage
\section{Unsupervised Anomaly detection with recency}
\newpage
\section{Usage of RBF loss function}
\newpage
\section{Experiments}
\newpage
\subsection{Data}
\subsubsection{Numenta Anomaly Benchmark (NAb)}
\newpage
\section{Execution and Results}
\newpage
\section{Discussion}
\newpage
\section{Experiment Infrastructure}
\newpage
\subsection{Experiment Management using MLflow}
\newpage
\subsection{Parallel execution using Docker}
\newpage
\section{Best practices}
Following steps were taken to maximize the efficiency and speed of research:
\begin{enumerate}
	\item Use version control to track the code and share between different devices.
	\item Separate code from data. This will keep the code base small and easy to debug.
	\item Separate input data,working data and output data.
	\begin{itemize}
		\item \textbf{Input Data:} Input data-set that never change. For my case it is NAB and other external datasets.
		\item \textbf{Working Data:} nothing for now.
		\item \textbf{Output Data:} Results and threshold profiles in my case. 
	\end{itemize}
	\item Separate options from parameter. This is important:
 	\begin{itemize}
 		\item Options specify how your algorithm should run. For example data path, working directory and result directory path, epochs, learning rate and so on.
 		\item parameters are the result of training data. it includes the score and hyper-parameters. 
 	\end{itemize}
	
\end{enumerate}
\newpage
\subsection{Moving from jupyterlab to pycharm}
While working with jupyterlab notebook following routine was followed:
1- Load data with sample function
2- Write an algorithm
3- Test the results
4- Write general executeable .py file.
5- Get results on server

Since we needed to track change on two different places,
it was becoming harder to track the bugs and improve on efficiency.
That's why pycharm was selected to create executeable files and test
algorithms at the same time.
\section{Reference Usage}
\newpage
\section{References}
\begingroup
\nocite{*}
\renewcommand{\section}[2]{}
\bibliographystyle{IEEEtrans}`
\bibliography{third}
\endgroup
\end{document}
